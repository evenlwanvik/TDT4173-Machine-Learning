%%% Template originaly created by Karol Kozioł (mail@karol-koziol.net) and modified for ShareLaTeX use

\documentclass[a4paper,11pt]{article}

\usepackage[T1]{fontenc}
\usepackage[utf8]{inputenc}
\usepackage{graphicx}
\usepackage{xcolor}

\renewcommand\familydefault{\sfdefault}
\usepackage{tgheros}
\usepackage[defaultmono]{droidmono}

\usepackage{amsmath,amssymb,amsthm,textcomp}
\usepackage{enumerate}
\usepackage{multicol}
\usepackage{tikz}

\usepackage{geometry}
\geometry{total={210mm,297mm},
left=25mm,right=25mm,%
bindingoffset=0mm, top=20mm,bottom=20mm}


\linespread{1.3}

\newcommand{\linia}{\rule{\linewidth}{0.5pt}}

% custom theorems if needed
\newtheoremstyle{mytheor}
    {1ex}{1ex}{\normalfont}{0pt}{\scshape}{.}{1ex}
    {{\thmname{#1 }}{\thmnumber{#2}}{\thmnote{ (#3)}}}

\theoremstyle{mytheor}
\newtheorem{defi}{Definition}

% my own titles
\makeatletter
\renewcommand{\maketitle}{
\begin{center}
\vspace{2ex}
{\huge \textsc{\@title}}
\vspace{1ex}
\\
\linia\\
\@author \hfill \@date
\vspace{4ex}
\end{center}
}
\makeatother
%%%

% custom footers and headers
\usepackage{fancyhdr}
\pagestyle{fancy}
\lhead{}
\chead{}
\rhead{}
\lfoot{Assignment \textnumero{} 5}
\cfoot{}
\rfoot{Page \thepage}
\renewcommand{\headrulewidth}{0pt}
\renewcommand{\footrulewidth}{0pt}
%

% code listing settings
\usepackage{listings}
\lstset{
    language=Python,
    basicstyle=\ttfamily\small,
    aboveskip={1.0\baselineskip},
    belowskip={1.0\baselineskip},
    columns=fixed,
    extendedchars=true,
    breaklines=true,
    tabsize=4,
    prebreak=\raisebox{0ex}[0ex][0ex]{\ensuremath{\hookleftarrow}},
    frame=lines,
    showtabs=false,
    showspaces=false,
    showstringspaces=false,
    keywordstyle=\color[rgb]{0.627,0.126,0.941},
    commentstyle=\color[rgb]{0.133,0.545,0.133},
    stringstyle=\color[rgb]{01,0,0},
    numbers=left,
    numberstyle=\small,
    stepnumber=1,
    numbersep=10pt,
    captionpos=t,
    escapeinside={\%*}{*)}
}

%%%----------%%%----------%%%----------%%%----------%%%

\begin{document}

\title{Machine Learning Assignment 1}

\author{Even Wanvik, NTNU}

\date{28/08/2019}

\maketitle

\section{Theory}

\subsection{What is \textit{concept learning}? + explain with an example}
\begin{quotation}
  ``The problem of searching through a predefined space of potential hyptheses for the hypothesis that best fits the training examples''
  \em Tom Michell
\end{quotation}
When a human being is learning something, much of it is based on generalized concepts gained from past experiences. For instance, if a human were to identify if a certain type of car, we differentiate the betwwen the cars based on a set of features. This set of features is what can be called a concept. 

Similarly, we can provide a machine with a training sample of a given signal or dataset from which it can learn the correct concepts needed to identify wether new data or objects belong to a specific category. These generalized concepts is commonly referred to as a hypothesis. 

\textbf{An example:} Let's say we want to identify reptiles in a dataset containing all types of animals. We extract a random subset for training the model, in which we have a set of features; \textit{scales, coldBlooded, legs, eggLaying}. To start with we have a random sample from the training set as the starting hypothesis. This hypothesis will constantly evolve when we constantly challenge the current hypothesis against the training data. This will go on until the hypothesis remains unchanged, and we have the best possible concept needed to differentiate reptiles from other animals.


\subsection{What is function approximation and why do we need them?}
Function approximation is the process of adjusting the given model, or function, to most likely represent the true target function. As for the evolution of hypothesis explained in the previous task, we need these function approximations to actively determine the vital parameters and their weight.

\newpage
\subsection{What is inductive bias in the context of machine learning, and why is it so important? Decision tree learning and the candidate elimination algorithm are two different learning algorithms. What can you say about the inductive bias for each of them?}
\begin{quotation}
  ``An inductive bias of a learner is the set of additional assumptions sufficient to justify its inductive inerference as deductive inference''
  \em Tom Michell
\end{quotation}
An inductive bias provide the concept learning algorithm with a method of favoring certain learning models over others, i.e. given a data set, which learning model (told by the inductive bias) should be chosen.






% code from http://rosettacode.org/wiki/Fibonacci_sequence#Python
\begin{lstlisting}[label={list:first},caption=Sample Python code -- Fibonacci sequence calculated analytically.]
from math import *

# define function 
def analytic_fibonacci(n):
  sqrt_5 = sqrt(5);
  p = (1 + sqrt_5) / 2;
  q = 1/p;
  return int( (p**n + q**n) / sqrt_5 + 0.5 )
 
# define range
for i in range(1,31):
  print analytic_fibonacci(i)
\end{lstlisting}

Following Listing~\ref{list:first}\ldots{} 
Lorem ipsum dolor sit amet, consectetur adipiscing elit, sed do eiusmod tempor incididunt ut labore et dolore magna aliqua. Ut enim ad minim veniam, quis nostrud exercitation ullamco laboris nisi ut aliquip ex ea commodo consequat. Duis aute irure dolor in reprehenderit in voluptate velit esse cillum dolore eu fugiat nulla pariatur. Excepteur sint occaecat cupidatat non ident, sunt in culpa qui officia deserunt mollit anim id est laborum.

\section*{blem 2}

\begin{lstlisting}[label={list:second},caption=Sample Bash code.]
#! /bin/bash
python stage1.py
echo "Stage I done!"
python stage2.py
echo "Stage II done!"
python stage3.py
echo "Stage III done!"
\end{lstlisting}

Lorem ipsum dolor sit amet, consectetur adipiscing elit, sed do eiusmod tempor incididunt ut labore et dolore magna aliqua. Ut enim ad minim veniam, quis nostrud exercitation ullamco laboris nisi ut aliquip ex ea commodo consequat. Duis aute irure dolor in reprehenderit in voluptate velit esse cillum dolore eu fugiat nulla pariatur. Excepteur sint occaecat cupidatat non ident, sunt in culpa qui officia deserunt mollit anim id est laborum.

\end{document}
